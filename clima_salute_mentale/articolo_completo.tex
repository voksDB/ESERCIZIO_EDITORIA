% Options for packages loaded elsewhere
\PassOptionsToPackage{unicode}{hyperref}
\PassOptionsToPackage{hyphens}{url}
%
\documentclass[
]{article}
\usepackage{amsmath,amssymb}
\usepackage{iftex}
\ifPDFTeX
  \usepackage[T1]{fontenc}
  \usepackage[utf8]{inputenc}
  \usepackage{textcomp} % provide euro and other symbols
\else % if luatex or xetex
  \usepackage{unicode-math} % this also loads fontspec
  \defaultfontfeatures{Scale=MatchLowercase}
  \defaultfontfeatures[\rmfamily]{Ligatures=TeX,Scale=1}
\fi
\usepackage{lmodern}
\ifPDFTeX\else
  % xetex/luatex font selection
\fi
% Use upquote if available, for straight quotes in verbatim environments
\IfFileExists{upquote.sty}{\usepackage{upquote}}{}
\IfFileExists{microtype.sty}{% use microtype if available
  \usepackage[]{microtype}
  \UseMicrotypeSet[protrusion]{basicmath} % disable protrusion for tt fonts
}{}
\makeatletter
\@ifundefined{KOMAClassName}{% if non-KOMA class
  \IfFileExists{parskip.sty}{%
    \usepackage{parskip}
  }{% else
    \setlength{\parindent}{0pt}
    \setlength{\parskip}{6pt plus 2pt minus 1pt}}
}{% if KOMA class
  \KOMAoptions{parskip=half}}
\makeatother
\usepackage{xcolor}
\setlength{\emergencystretch}{3em} % prevent overfull lines
\providecommand{\tightlist}{%
  \setlength{\itemsep}{0pt}\setlength{\parskip}{0pt}}
\setcounter{secnumdepth}{-\maxdimen} % remove section numbering
% definitions for citeproc citations
\NewDocumentCommand\citeproctext{}{}
\NewDocumentCommand\citeproc{mm}{%
  \begingroup\def\citeproctext{#2}\cite{#1}\endgroup}
\makeatletter
 % allow citations to break across lines
 \let\@cite@ofmt\@firstofone
 % avoid brackets around text for \cite:
 \def\@biblabel#1{}
 \def\@cite#1#2{{#1\if@tempswa , #2\fi}}
\makeatother
\newlength{\cslhangindent}
\setlength{\cslhangindent}{1.5em}
\newlength{\csllabelwidth}
\setlength{\csllabelwidth}{3em}
\newenvironment{CSLReferences}[2] % #1 hanging-indent, #2 entry-spacing
 {\begin{list}{}{%
  \setlength{\itemindent}{0pt}
  \setlength{\leftmargin}{0pt}
  \setlength{\parsep}{0pt}
  % turn on hanging indent if param 1 is 1
  \ifodd #1
   \setlength{\leftmargin}{\cslhangindent}
   \setlength{\itemindent}{-1\cslhangindent}
  \fi
  % set entry spacing
  \setlength{\itemsep}{#2\baselineskip}}}
 {\end{list}}
\usepackage{calc}
\newcommand{\CSLBlock}[1]{\hfill\break\parbox[t]{\linewidth}{\strut\ignorespaces#1\strut}}
\newcommand{\CSLLeftMargin}[1]{\parbox[t]{\csllabelwidth}{\strut#1\strut}}
\newcommand{\CSLRightInline}[1]{\parbox[t]{\linewidth - \csllabelwidth}{\strut#1\strut}}
\newcommand{\CSLIndent}[1]{\hspace{\cslhangindent}#1}
\ifLuaTeX
\usepackage[bidi=basic]{babel}
\else
\usepackage[bidi=default]{babel}
\fi
\babelprovide[main,import]{italian}
% get rid of language-specific shorthands (see #6817):
\let\LanguageShortHands\languageshorthands
\def\languageshorthands#1{}
\ifLuaTeX
  \usepackage{selnolig}  % disable illegal ligatures
\fi
\usepackage{bookmark}
\IfFileExists{xurl.sty}{\usepackage{xurl}}{} % add URL line breaks if available
\urlstyle{same}
\hypersetup{
  pdftitle={Salute mentale e cambiamento climatico: impatti e strategie di adattamento},
  pdfauthor={Damon Bianchi},
  pdflang={it},
  hidelinks,
  pdfcreator={LaTeX via pandoc}}

\title{Salute mentale e cambiamento climatico: impatti e strategie di
adattamento}
\author{Damon Bianchi}
\date{2025-07-04}

\begin{document}
\maketitle
\begin{abstract}
Il cambiamento climatico influisce sulla salute mentale attraverso
stress ambientale, insicurezza alimentare e condizioni socio-economiche
precarie. questo articolo presenta tre studi principali: 1. Navigating
psychosocial dimensions, che esamina come le strategie di adattamento
influenzino il benessere. 2. Strategies for resilience, che propone
interventi per mitigare fame e salute mentale. 3. Locating the built
environment, una revisione globale sull'impatto dell'ambiente costruito
sulla salute mentale.
\end{abstract}

\subsection{Il ruolo dell'ambiente costruito nei modelli empirici su
cambiamento climatico e salute
mentale}\label{il-ruolo-dellambiente-costruito-nei-modelli-empirici-su-cambiamento-climatico-e-salute-mentale}

📄 Articolo originale:
\href{https://www.sciencedirect.com/science/article/pii/S1877343524000800}{Navigating
Psychosocial Dimensions}

Il cambiamento climatico rappresenta una sfida complessa che va oltre
gli effetti ambientali, toccando profondamente la salute mentale e il
benessere delle persone. L'articolo di Stacey C. Heath esplora come le
strategie di adattamento adottate dalle comunità e dagli individui
possano influenzare non solo la capacità di far fronte agli impatti
climatici, ma anche lo stato psicologico e sociale delle persone
coinvolte.

Gli eventi climatici estremi, la perdita di risorse naturali e le
trasformazioni socioeconomiche collegate al cambiamento climatico
possono generare stress prolungato, ansia e senso di incertezza. Heath
sottolinea che un approccio efficace all'adattamento deve quindi
integrare non solo misure tecniche o infrastrutturali, ma anche
interventi che rafforzino la resilienza psicologica e il supporto
sociale.

In particolare, l'autrice invita a superare la tradizionale separazione
tra strategie ambientali e benessere umano, promuovendo una visione
integrata che tenga conto delle interconnessioni tra condizioni
ambientali, fattori sociali e dimensioni psicosociali. Solo attraverso
questa prospettiva olistica è possibile sviluppare politiche e programmi
che migliorino la qualità della vita nelle comunità più vulnerabili,
garantendo un benessere sostenibile nel lungo termine.

In sintesi, l'articolo mette in luce l'importanza di considerare le
dimensioni emotive e sociali nel progettare strategie di adattamento al
cambiamento climatico, affinché queste possano contribuire efficacemente
a mitigare gli effetti negativi sulla salute mentale e promuovere la
resilienza collettiva.

(Heath 2025)

\begin{center}\rule{0.5\linewidth}{0.5pt}\end{center}

\subsection{Strategie di resilienza per mitigare gli effetti del
cambiamento climatico sulla salute
mentale}\label{strategie-di-resilienza-per-mitigare-gli-effetti-del-cambiamento-climatico-sulla-salute-mentale}

📄 Articolo originale:
\href{https://www.sciencedirect.com/science/article/pii/S2666154325003941}{Strategies
for Resilience}

Il cambiamento climatico non influisce solo sull'ambiente, ma ha
conseguenze dirette e indirette sulla sicurezza alimentare e sulla
salute mentale delle popolazioni più vulnerabili. Questo articolo
analizza le strategie di resilienza volte a ridurre l'impatto della
crisi climatica su fame e benessere psicologico.

Gli autori evidenziano come la crescente insicurezza alimentare, dovuta
a eventi climatici estremi e alla perdita di produttività agricola,
aggravi lo stress psicologico, aumentando i rischi di ansia, depressione
e altre problematiche legate alla salute mentale. La fame e la
malnutrizione diventano quindi fattori di rischio per il benessere
mentale, creando un circolo vizioso difficile da interrompere.

Per contrastare questi effetti, l'articolo propone un approccio
multidimensionale che include interventi mirati a: - migliorare la
sicurezza alimentare attraverso pratiche agricole sostenibili e
diversificate; - rafforzare le reti di supporto sociale e comunitario; -
promuovere programmi di assistenza psicologica nelle aree colpite da
crisi alimentari e climatiche.

In conclusione, la resilienza efficace richiede politiche integrate che
affrontino simultaneamente le sfide ambientali, alimentari e
psicosociali, al fine di tutelare sia la sopravvivenza fisica sia il
benessere mentale delle comunità esposte ai cambiamenti climatici.

(Mezieobi et al. 2025)

\begin{center}\rule{0.5\linewidth}{0.5pt}\end{center}

\subsection{Impatto psicologico del cambiamento climatico: strategie di
adattamento e
benessere}\label{impatto-psicologico-del-cambiamento-climatico-strategie-di-adattamento-e-benessere}

📄 Articolo originale:
\href{https://www.sciencedirect.com/science/article/pii/S1877343524000800}{Locating
the built environment}

Questo articolo presenta il protocollo per una revisione sistematica
volta a mappare come l'ambiente costruito --- ovvero gli spazi fisici e
le infrastrutture create dall'uomo --- sia integrato nei modelli
empirici che studiano le relazioni tra cambiamento climatico e salute
mentale.

Il lavoro si propone di analizzare la letteratura scientifica globale
per identificare le modalità con cui fattori legati all'ambiente
costruito, come la progettazione urbana, la qualità degli edifici e la
pianificazione territoriale, influenzano gli impatti psicologici e
psicosociali associati ai cambiamenti climatici.

L'obiettivo è fornire una base conoscitiva che aiuti a comprendere come
l'adattamento climatico e la mitigazione possano essere supportati da
interventi sul built environment per migliorare il benessere mentale
nelle comunità più vulnerabili.

Questa revisione permetterà di evidenziare lacune e potenziali aree di
sviluppo nella ricerca, facilitando la creazione di strategie integrate
che considerino sia i fattori ambientali che quelli sociali nella lotta
agli effetti negativi del cambiamento climatico sulla salute mentale.

(McCall et al. 2025)

\begin{center}\rule{0.5\linewidth}{0.5pt}\end{center}

\phantomsection\label{refs}
\begin{CSLReferences}{1}{0}
\bibitem[\citeproctext]{ref-heath2025navigating}
Heath, Stacey C. 2025. {«Navigating psychosocial dimensions:
understanding the intersections of adaptation strategies and well-being
outcomes in the context of climate change»}. \emph{Current Opinion in
Environmental Sustainability} 72: 101493.

\bibitem[\citeproctext]{ref-mccall2025locating}
McCall, Hannah, James McGowan, Sarah O'Connor, John McGowan, e Brian
O'Neill. 2025. {«Locating the built environment within existing
empirical models of climate change and mental health: protocol for a
global systematic scoping review»}. \emph{BMJ Open} 15 (4): e093222.

\bibitem[\citeproctext]{ref-mezieobi2025strategies}
Mezieobi, Kelechi Chinemerem, Esther Ugo Alum, Chukwuoyims Kevin Egwu,
Daniel Ejim Uti, Benedict Nnachi Alum, Ainebyoona Christine, e Omuna
Daniel. 2025. {«Strategies for Resilience: Mitigating the Effects of
Climate Change on Hunger and Mental Health»}. \emph{Journal of
Agriculture and Food Research}, 102023.

\end{CSLReferences}

\end{document}
